\*this post is partly funded by EF Academic Grant FY23-1030
(\[Sandra\](https://twitter.com/sandJohnson),
\[Kerrie\](https://twitter.com/KerrieMengersen) and
\[Patrick\](https://twitter.com/Path_doc)) with special thanks to
\[Anders Madsen\](https://vbn.aau.dk/en/persons/100976) for his
contribution to the interactive version of the BN model, and
constructive feedback and discussion with
\[Barnabé\](https://twitter.com/barnabemonnot),
\[Ben\](https://twitter.com/benjaminion_xyz),
\[Mike\](https://twitter.com/mikeneuder),
\[Mikhail\](https://twitter.com/mkalinin2),
\[Roberto\](https://twitter.com/robsaltini)\*.

\[EIP-7251\](https://eips.ethereum.org/EIPS/eip-7251) proposes
increasing the \*MAX_EFFECTIVE_BALANCE\* constant from 32 ETH to 2,048
ETH, but the \*minimum effective balance\* required to join as a
validator remains unchanged at 32 ETH.

The expectation is that \"Proposer selection is already weighted by the
ratio of their effective balance to MAX_EFFECTIVE_BALANCE. Due to the
lower probabilities, this change will slightly increase the time it
takes to calculate the next proposer index.\"

We undertook some additional analysis of proposer selection which is
summarised in this blog post.

\## Proposer selection process

Proposer selection is a two-stage process: 1. Being \*selected as the
candidate\* from the list of shuffled validator indices. 2. Passing the
\*proposer eligibility\* check

\## The swap-or-not-shuffle technique \[\[1\]\](#first) is used to
shuffle the validator indices in preparation for the selection of a
block proposer.

This is done in
\[compute_shuffled_index\](https://github.com/ethereum/consensus-specs/blob/9c35b7384e78da643f51f9936c578da7d04db698/specs/phase0/beacon-chain.md#compute_shuffled_index).
The computation to determine the proposer for the next block is done in
\[compute_proposer_index\](https://github.com/ethereum/consensus-specs/blob/9c35b7384e78da643f51f9936c578da7d04db698/specs/phase0/beacon-chain.md#compute_proposer_index)
(shown below):

    python def compute_proposer_index(state: BeaconState, indices:
Sequence\[ValidatorIndex\], seed: Bytes32) -\> ValidatorIndex: \"\"\"
Return from ''indices'' a random index sampled by effective balance.
\"\"\" assert len(indices) \> 0 MAX_RANDOM_BYTE = 2\*\*8 - 1 i =
uint64(0) total = uint64(len(indices)) while True: candidate_index =
indices\[compute_shuffled_index(i random_byte = hash(seed +
uint_to_bytes(uint64(i // 32)))\[i effective_balance =
state.validators\[candidate_index\].effective_balance if
effective_balance \* MAX_RANDOM_BYTE \>= MAX_EFFECTIVE_BALANCE \*
random_byte: return candidate_index i += 1    

Therefore, we iterate through the shuffled indices, starting with the
first entry and then check whether it passes the selection criteria. If
it doesn't, then the next validator index in the array goes through the
same check.

As we can see from the code, the validator's effective balance (EB) is
multiplied by 255 (i.e. $MAX\_RANDOM\_BYTE = 2^8 - 1 = 255$) and then
compared to the product of the generated $random\_byte$ $(r)$ and
$MAX\_EFFECTIVE\_BALANCE = 2,048 ETH$.

Figure 1 (a) is an exposition of random byte values generated for
716,800 validators from the \*random_byte\* assignment statement below.
Superimposed on the histogram of these random byte integers is a uniform
distribution. As expected, the random bytes appear to visually resemble
values drawn from a uniform distribution: $r \sim U(0,255)$. We confirm
this in the Q--Q plot in Figure 1 (b). Hence we can assume in subsequent
calculations that the random bytes, $r$, have a uniform distribution,
$r \sim U(0,255)$.

    python random_byte = hash(seed + uint_to_bytes(uint64(i // 32)))\[i
    \<img src=\"upload://qqt3TqttDP5dBFg8SjSsUOQlGt0.png\" width=\"340\"
height=\"355\"\>\<img src=\"upload://uSaQN1GdZw73WfZAuRrvqOfG1HS.png\"
width=\"340\" height=\"340\"\> \<br\>
&nbsp;&nbsp;&nbsp;&nbsp;&nbsp;&nbsp;&nbsp;&nbsp;&nbsp;&nbsp;&nbsp;&nbsp;&nbsp;&nbsp;&nbsp;&nbsp;&nbsp;&nbsp;&nbsp;&nbsp;&nbsp;&nbsp;&nbsp;&nbsp;&nbsp;&nbsp;&nbsp;&nbsp;&nbsp;&nbsp;&nbsp;&nbsp;&nbsp;&nbsp;&nbsp;&nbsp;&nbsp;&nbsp;&nbsp;&nbsp;&nbsp;&nbsp;&nbsp;&nbsp;&nbsp;&nbsp;&nbsp;&nbsp;&nbsp;&nbsp;&nbsp;&nbsp;(a)&nbsp;&nbsp;&nbsp;&nbsp;&nbsp;&nbsp;&nbsp;&nbsp;&nbsp;&nbsp;&nbsp;&nbsp;
&nbsp;&nbsp;&nbsp;&nbsp;&nbsp;&nbsp;&nbsp;&nbsp;&nbsp;&nbsp;&nbsp;&nbsp;&nbsp;&nbsp;
&nbsp;&nbsp;&nbsp;&nbsp;&nbsp;&nbsp;&nbsp;&nbsp;&nbsp;&nbsp;&nbsp;&nbsp;&nbsp;&nbsp;
&nbsp;&nbsp;&nbsp;&nbsp;&nbsp;&nbsp;&nbsp;&nbsp;&nbsp;&nbsp;&nbsp;&nbsp;&nbsp;&nbsp;
&nbsp;&nbsp;&nbsp;&nbsp;&nbsp;&nbsp;&nbsp;&nbsp;&nbsp;&nbsp;&nbsp;&nbsp;&nbsp;&nbsp;(b)

\*Figure 1: (a) Distribution of 716,800 random bytes generated from the
spec (b) Quantile-quantile plot of the generated random bytes against a
uniform distribution - U(0,255)\*

The probability of a validator being the proposer if their index was
selected from the list is calculated as follows (where $\therefore$
means \*therefore\*):

$P(proposer \texttt{ } check \texttt{ }passed) = P(EB * 255 \geqslant MaxEB * r)$,
where $r \sim U(0,255)$, $r = random\_byte$,
$EB = validator\_e$\*ff\*$ective\_balance$\<br\>
$\therefore P(proposer \texttt{ } check \texttt{ }passed) = P\left( r \leqslant  \frac{255*EB}{MaxEB}\right)$
\<br\>
$\therefore \textit{if } EB = MaxEB \implies P(proposer \texttt{ } check \texttt{ }passed) = 1$

In other words, if the effective balance of the candidate validator
equalled the maximum effective balance, then the validator becomes the
proposer with probability 1.

---

When the maximum effective balance is increased to 2,048 ETH, the
probability of passing the proposer eligibility test will vary depending
on the extent of validator consolidation.

As before, a fully consolidated validator with an EB of \*\*2, 048 ETH
(64 \* 32 ETH)\*\* will become a proposer with a probability of 1 if the
validator's index was randomly selected as the next candidate.

However, if the randomly selected candidate validator has an EB of:

\*\*32 ETH\*\*:
$P(proposer \texttt{ } check \texttt{ }passed) = P\left(r \leqslant \frac{255*32}{2048} \right)$\<br\>
$\therefore P(proposer \texttt{ } check \texttt{ }passed) = P(r \leqslant 3.98) = \left(\frac{3.98-0}{255}\right) = 0.016$
\<br\>
$\therefore P(proposer \texttt{ } check \texttt{ }passed) \equiv \left(\frac{32}{2048}\right) = 0.016$
\<br\> \*\*64 ETH (2 \* 32 ETH)\*\*
$\therefore P(proposer \texttt{ } check \texttt{ } passed) = \left(\frac{64}{2048}\right) = 0.031$
\<br\> \*\*160 ETH (5 \* 32 ETH)\*\*
$\therefore P(proposer \texttt{ } check \texttt{ } passed) = \left(\frac{160}{2048}\right) = 0.078$
\<br\> \*\*320 ETH (10 \* 32 ETH)\*\*
$\therefore P(proposer \texttt{ } check \texttt{ }passed) = \left(\frac{320}{2048}\right) = 0.156$\<br\>
\*\*960 ETH (30 \* 32 ETH)\*\*
$\therefore P(proposer \texttt{ } check \texttt{ }passed) = \left(\frac{960}{2048}\right) = 0.469$

Figure 2 below depicts the increase in passing the proposer eligibility
test as the validator's effective balance increases.

&nbsp;&nbsp;&nbsp;&nbsp;&nbsp;&nbsp;&nbsp;&nbsp;&nbsp;&nbsp;&nbsp;&nbsp;&nbsp;&nbsp;&nbsp;&nbsp;&nbsp;&nbsp;&nbsp;&nbsp;&nbsp;&nbsp;&nbsp;&nbsp;&nbsp;&nbsp;&nbsp;&nbsp;&nbsp;&nbsp;&nbsp;&nbsp;&nbsp;&nbsp;\<img
src=\"upload://iHtiVZv1APGBghGV1Nia16WZJaT.png\" width=\"450\"
height=\"380\>̈ \*Figure 2: Probability of passing the proposer
eligibility check for a candidate validator with an EB ranging from 32
to 2,048 ETH\*\<br\> In other words, if a validator's index is selected
as the next candidate, the probability of passing the selection check to
propose the next block varies from 0.016 for an unconsolidated validator
(32ETH) to 1 for a fully consolidated validator (2,048ETH).

-----

Assuming a validator set size of 716,800, then the probability of any
validator, regardless of their effective balance, being first in the
list of shuffled validator indices (i.e. the candidate to be assessed as
the next block proposer) is: \<br\>
$P(candidate) = \frac{1}{(Active \texttt{ } validator \texttt{ } set \texttt{ } size)} = \frac{1}{ 716,800} = 1.395*10^{-6}$.

Currently (i.e. prior to EIP-7521), providing a validator maintains its
effective balance at 32 ETH, once its index is selected as the next
candidate to be checked, it would pass the proposer selection test with
certainty (i.e. probability of $1$, or $100\%$).

Putting it another way:\<br\> \*Given EB=32 ETH\*, then the probability
of being selected as the $1^{st}$ candidate and becoming the next
proposer is calculated as follows:\<br\> $P(candidate \texttt{ }$ $\&$
$\texttt{ }proposer$ $check$ $passed)$ \<br\>
$= P(candidate) * P(proposer$ $check$ $passed / candidate)$ \<br\>
$= P(candidate) * P(proposer$ $check$ $passed)$ &nbsp; &nbsp; &nbsp;
&nbsp; (since these two events are independent) \<br\>
$= \left(\frac{1}{716800}\right) * 1 \approx 1.395 * 10^{-6}$

After MaxEB = 2,048 ETH, this changes to:\<br\> $P(candidate$ $\&$
$proposer$ $check$ $passed) =$
$\left(\frac{1}{716800}\right) * \left(\frac{32}{2048}\right)$
$\approx 2.18*10^{-8}$

For the current MaxEB of 32 ETH, we know that if the validator is
selected as the $1^{st}$ candidate in the shuffled index, then they will
definitely be the proposer for the next block.

When we consider the increased MaxEB proposal, then it may be quite
likely that a validator is selected as the next candidate validator for
the proposer eligibility test at subsequent rounds (e.g.
$2^{nd}, 3^{rd}, ..., n^{th}$), because a selected candidate will be
less likely to pass the proposer check.

Therefore we calculate the probability that a 32 ETH validator, $v$, is
selected at round $i$ given that all the previous rounds
$(1,2, ... (i-1))$ were unsuccessful, i.e. we sum over all the possible
rounds that this validator may have been selected.

For example, if validator $v$ is selected at round 3, there would have
been a rejection of another validator at round 1 and round 2. Stating
this more generally: \<br\>
$P(validator \texttt{ } v \texttt{ } selected \texttt{ } as \texttt{ } next \texttt{ } proposer) = \sum_{i=1}^{716,800} p_i * \left( \prod_{j=1}^i (1-p_{j-1}) \right), \texttt{ } where$
\<br\>
$p_i = P(round \texttt{ }i \texttt{ } proposer) = P(round \texttt{ } i \texttt{ } candidate) * P(passing \texttt{ } proposer \texttt{ } eligibility)$,
$and 
\texttt{ } p_0 = 0$ \<br\> This probability is also $1.395 * 10^{-6}$.

In summary, with an increased MaxEB, a solo staker would be selected as
the next proposer with a probability of $2.18*10^{-8}$, from the current
$1.395 * 10^{-6}$ probability if they are the \*\*first\*\* candidate in
the shuffled index. However, their probability over all the possible
outcomes will be the same as is currently the case, if all the other
validators have the same EB and hence the same probability of passing
the proposer check.

It gets a bit more complex when the minimal solo staker is competing
against validators that are more likely to pass the proposer eligibility
check, e.g. after implementation of EIP-7521.

The probability calculations above assumes a validator set with no
consolidation, i.e. the validator set size remained unchanged with the
introduction of EIP-7521. In practice this is highly unlikely.
Therefore, we consider an example scenario.

\### Example Scenario: \*Active validator set has validators with
varying EBs (i.e. stakers may choose different strategies for validator
consolidation)\* To the best of our knowledge, the analyses to date have
mainly been for a homogeneous validator set, i.e. all validators have
the same effective balance, either unconsolidated (32 ETH) or fully
consolidated (2,048 ETH). This includes the analysis we conducted above.

Hence, a more interesting scenario would be an active validator set with
a variety of effective balances.

Therefore, let us assume that we have an active validator set with
716,800 validators prior to EIP-7251. The stakers in this validator set
choose to combine some of their validators to a greater or lesser
extent, and may in fact leave several of their validators
unconsolidated, i.e. at 32 ETH.

We visualise this hypothetical consolidation of the validator set in
Figure 3 below.

We group the stakers into five categories: - Small-scale solo stakers
(32 - a few hundred ETH) - Large-scale individual solo stakers (1000+
ETH) - Large-scale institutional solo stakers (ie. companies staking
their own ETH) - Centralised staking pools - Semi-decentralised staking
pools (e.g. Rocketpool, Lido\... )

\![\](upload://lNZenxS8kjiAuO8A48qvOw8JoDg.png)
&nbsp;&nbsp;&nbsp;&nbsp;&nbsp;&nbsp;&nbsp;&nbsp;&nbsp;&nbsp;&nbsp;&nbsp;&nbsp;&nbsp;&nbsp;&nbsp;&nbsp;&nbsp;\*Figure
3: Visual representation of the example scenario for validator
consolidation\* \<br\>

We put a potential distribution over the various validator
consolidations in our scenario for illustrative purposes. The
distributions can be adjusted as required.

The \*Validator numbers after consolidation\* row (green) shows the
reduced validator numbers for each category, using the consolidations
shown in each column (yellow).

Based on the chosen configuration, the total validator set size reduces
to 329,810 (last value in the diagram).

\##

\### Bayesian network for proposer selection We built a simple Bayesian
network (BN) to illustrate the dependencies between the different nodes
in the BN model for the example scenario described above (Figures 4 and
5).

A Bayesian network (BN) is an acyclic directed graph of nodes and edges.
The nodes represent the key factors of the system or problem being
modelled, and the edges between the nodes indicate dependencies.
Uncertainty and the stength of the dependencies between connected nodes
are explicitly captured in the node probability tables that are attached
to each node in the model. An object-oriented version of a BN (OOBN) may
be used to make BNs less cluttered and more readible, by grouping
related nodes and processes in OOBN submodels. \[\[2\]\](#second)

\<img src=\"upload://uUDZCAzVzYTp8fdEWGhKPAvh4QL.png\" width=\"235\"
height=\"300\"\>&nbsp; &nbsp; \<img
src=\"upload://eHb9l4uK27PcsufEVDIULFj59hM.png\" width=\"430\"\> \<br\>
\*Figure 4: OOBN for proposer
selection\*&nbsp;&nbsp;&nbsp;&nbsp;&nbsp;&nbsp;&nbsp;&nbsp;&nbsp;&nbsp;&nbsp;&nbsp;&nbsp;&nbsp;&nbsp;&nbsp;&nbsp;&nbsp;&nbsp;&nbsp;
\*Figure 5: OOBN subnet for probability calculations\*

We calculate the probability that a validator is selected and passes the
check for proposer eligibility, as the product of the probability of
being the candidate index and the probability of passing the proposer
check, since these two events are independent, i.e.

$P(A \cap B) = P(A/B)P(B) = P(A)P(B)$.

The probabilities shown in Figure 2 above are used in the node
probability table for BN node \*Proposer check\*, which captures the
probability that a validator will pass the proposer eligibility check
for the various consolidated validator sizes.

Using the logic from the equation above and the previous calculations
for the probability that a validator passes the eligibility check, we
can now determine: - the probability for a validator from each
consolidation group (e.g. single, 2-fold, etc) to be selected as a
candidate validator from the shuffled validator indices, and - the
probability that this candidate type will pass the eligibility check for
being a proposer.

These probabilities are shown in Table 1 below in the last two columns,
and are used to populate the probability node tables for BN nodes
\*Validator type selected as candidate for proposer duty\* and
\*Proposer for next block\*, respectively.

For the example scenario described above and assuming a validator set
size of 716,800, we can use the BN model to gain some insights into the
probabilities of different 'types' of validators. By 'type' we mean the
extent to which the validator has been consolidated.

Running this model generates the marginal probabilities shown in Figure
6(a). The proportion of validator categories in the reduced validator
set are visible in BN node \*Consolidated validator types\* and are in
the second column of Table 1.

We see that based on the configuration of the various types of
validators, the active validator set reduced to 329,810.

\### Table 1: Probability that validator type is selected as a candidate
proposer \| \*\*Type of\*\* \<br\>
\*\*validator\*\*\<br\>\*\*consolidation\*\* \| \*\*Proportion\*\*
\<br\>\*\*of validators\*\* \<br\> \*\*consolidating\*\* \|\*\*Number
of\*\* \<br\> \*\*validators in\*\* \<br\> \*\*consolidated\*\* \<br\>
\*\*validator set\*\*\|\*\*P(validator type\*\* \<br\> \*\*selected
as\*\* \<br\> \*\*candidate)\*\* \|\*\*P(selected
validator\*\*\<br\>\*\*type is next\*\* \<br\> \*\*proposer)\*\* \|
\|:----------------:\|:----------------:\|:------------:\|:----------------:\|:----------------:\|
\| Single\| 28.75 \| Partial (2-fold)\| 25.75 \| Partial (5-fold)\|
15.00 \| Partial (10-fold)\| 9.00 \| Partial (30-fold)\| 8.50 \| Full
(64-fold)\| 13.00 \|\*\*TOTAL\*\*\| \*\*100.00

\<img src=\"upload://12f7eIF4DLSw9dK1nyOQdIDBjX5.png\" width=\"200\"
height=\"300\"\>&nbsp;&nbsp;&nbsp;&nbsp;&nbsp;&nbsp;&nbsp;&nbsp;&nbsp;&nbsp;\<img
src=\"upload://wrYcCqC3ZsqnJK5DnBgMa7YouiT.png\" width=\"200\"
height=\"300\"\>&nbsp;&nbsp;&nbsp;&nbsp;&nbsp;&nbsp;&nbsp;&nbsp;&nbsp;&nbsp;\<img
src=\"upload://xAOjVBPK0ha7wc8oUmSpdOtgRxP.png\" width=\"200\"
height=\"300\"\> \<br\>
&nbsp;&nbsp;&nbsp;&nbsp;&nbsp;&nbsp;&nbsp;&nbsp;&nbsp;&nbsp;&nbsp;&nbsp;&nbsp;&nbsp;&nbsp;&nbsp;&nbsp;&nbsp;&nbsp;&nbsp;&nbsp;&nbsp;&nbsp;&nbsp;&nbsp;&nbsp;&nbsp;&nbsp;&nbsp;&nbsp;&nbsp;&nbsp;(a)
&nbsp;&nbsp;&nbsp;&nbsp;&nbsp;&nbsp;&nbsp;&nbsp;&nbsp;&nbsp;&nbsp;&nbsp;
&nbsp;&nbsp;&nbsp;&nbsp;&nbsp;&nbsp;&nbsp;&nbsp;&nbsp;&nbsp;&nbsp;&nbsp;&nbsp;&nbsp;
&nbsp;&nbsp;&nbsp;&nbsp;&nbsp;&nbsp;&nbsp;&nbsp;&nbsp;&nbsp;&nbsp;&nbsp;&nbsp;&nbsp;
&nbsp;&nbsp;&nbsp;&nbsp;&nbsp;&nbsp;&nbsp;(b)&nbsp;&nbsp;&nbsp;&nbsp;&nbsp;&nbsp;&nbsp;&nbsp;&nbsp;&nbsp;&nbsp;&nbsp;&nbsp;&nbsp;
&nbsp;&nbsp;&nbsp;&nbsp;&nbsp;&nbsp;&nbsp;&nbsp;&nbsp;&nbsp;&nbsp;&nbsp;&nbsp;&nbsp;
&nbsp;&nbsp;&nbsp;&nbsp;&nbsp;&nbsp;&nbsp;&nbsp;&nbsp;&nbsp;&nbsp;&nbsp;&nbsp;&nbsp;
&nbsp;&nbsp;&nbsp;&nbsp;&nbsp;&nbsp;&nbsp;(c) \<br\>

\<img src=\"upload://mY48b4dkvt83EO4A1Jq5WHR2r0z.png\" width=\"200\"
height=\"300\"\>&nbsp;&nbsp;&nbsp;&nbsp;&nbsp;&nbsp;&nbsp;&nbsp;&nbsp;&nbsp;\<img
src=\"upload://5LRzB4k9Rxt81gegHerxjopptK2.png\" width=\"200\"
height=\"300\"\>&nbsp;&nbsp;&nbsp;&nbsp;&nbsp;&nbsp;&nbsp;&nbsp;&nbsp;&nbsp;\<img
src=\"upload://nBR4g3Nd6ya1pLJ3cIpVgcqTZxs.png\" width=\"200\"
height=\"300\>̈

\<br\>&nbsp;&nbsp;&nbsp;&nbsp;&nbsp;&nbsp;&nbsp;&nbsp;&nbsp;&nbsp;&nbsp;&nbsp;&nbsp;&nbsp;&nbsp;&nbsp;&nbsp;&nbsp;&nbsp;&nbsp;&nbsp;&nbsp;&nbsp;&nbsp;&nbsp;&nbsp;&nbsp;&nbsp;&nbsp;&nbsp;&nbsp;&nbsp;(d)
&nbsp;&nbsp;&nbsp;&nbsp;&nbsp;&nbsp;&nbsp;&nbsp;&nbsp;&nbsp;&nbsp;&nbsp;
&nbsp;&nbsp;&nbsp;&nbsp;&nbsp;&nbsp;&nbsp;&nbsp;&nbsp;&nbsp;&nbsp;&nbsp;&nbsp;&nbsp;
&nbsp;&nbsp;&nbsp;&nbsp;&nbsp;&nbsp;&nbsp;&nbsp;&nbsp;&nbsp;&nbsp;&nbsp;&nbsp;&nbsp;
&nbsp;&nbsp;&nbsp;&nbsp;&nbsp;&nbsp;&nbsp;(e)&nbsp;&nbsp;&nbsp;&nbsp;&nbsp;&nbsp;&nbsp;&nbsp;&nbsp;&nbsp;&nbsp;&nbsp;&nbsp;&nbsp;
&nbsp;&nbsp;&nbsp;&nbsp;&nbsp;&nbsp;&nbsp;&nbsp;&nbsp;&nbsp;&nbsp;&nbsp;&nbsp;&nbsp;
&nbsp;&nbsp;&nbsp;&nbsp;&nbsp;&nbsp;&nbsp;&nbsp;&nbsp;&nbsp;&nbsp;&nbsp;&nbsp;&nbsp;
&nbsp;&nbsp;&nbsp;&nbsp;&nbsp;&nbsp;&nbsp;(f)

\*Figure 6: Running proposer selection model for various scenarios\*

--- \### Running the OOBN model \#### Scenario 1: Running the model with
no evidence provided (a) The result of running the BN model is shown in
Figure 6(a). Therefore, assuming the active validator set consists of
the staking categories as described along with their respective example
consolidation strategies, the validator set will be reduced to 329,810
from 716,800. This adjusted validator set has different proportions of
consolidated validators as shown in node \*Consolidated validator
types\*. Across this validator set, the proportion of validators that
will pass the proposer test is 20.8

The probability of being the next proposer depends on being selected as
the next candidate and then passing the test. It appears
counter-intuitive that the probability is so much smaller than the
individual probabilities in the BN. The reason for this becomes clearer
when we look at the conditional probability table of this node:
\![NextProposerCPT\](upload://f6uyuaUWgf4396dOmytLrTa6ozi.png)

\#### Scenario 2: Running the model with all validators having 32 ETH
(b) If we enter evidence (shown as a state of a node being red) that the
validator set consists entirely of unconsolidated validators, and
therefore there is no reduction in the size of the validator set, the
probability of selecting an unconsolidated validator type is
understandably 100

\#### Scenario 3: Running the model with all validators having 64 ETH
(\*\*$\textbf{c}$\*\*) In this scenario we assume that all stakers
decided to consolidate their validators into 64 ETH validators. Here the
validator set size reduces to 358,400. As in Scenario 2, the first
selected validator type will be a validator with 64 ETH, i.e. 100

\#### Scenario 4: Running the model with all validators having MaxEB
(2,048 ETH) (d) In this scenario we assume that all stakers decided to
consolidate their validators to the maximum allowed, i.e. 2,048 ETH.
This is the largest reduction in validator set size, being just 11,200.
As in Scenarios 2 & 3, the first selected validator type will definitely
be a validator with an EB of MaxEB. For MaxEB the probability of passing
the proposer check is 3.12

\#### Scenario 5: Running the model assuming the validator set consists
only of small-scale stakers applying the example consolidation strategy
(e) The small staker group is assumed to mainly consist of validators
with 32 or 64 ETH, with a small proportion consolidated into 160 ETH
validators. Based on this strategy, the validator set reduces to
458,752. Across these small staker validators, the proportion that will
pass the proposer test is 3.44 38.75

\#### Scenario 6: assuming the validator set consists only of
semi-decentralized staking pools applying the example consolidation
strategy (f) In the example consolidation strategy for
semi-decentralized staking pools, they are assumed to have a fairly even
spread across the various extents of consolidation, with a slight
majority of single stake validators. Based on this strategy, the
validator set reduces to 312,853. Across this validator set, the
proportion of validators that will pass the proposer test is 28.13 26.11

---

\### Online version of Proposer Selection OOBN model We used \[HUGIN
EXPERT\](https://www.hugin.com/) when constructing the proposer
selection model. Hugin made the proposer selection OOBN model freely
available \[online\](https://demo.hugin.com/example/ProposerSelection)
with some widgets to allow interaction with the model.

\##

\## Increased proposer selection time As pointed out in the ethresear.ch
post \[Increase the MAX_EFFECTIVE_BALANCE -- a modest
proposal\](https://ethresear.ch/t/increase-the-max-effective-balance-a-modest-proposal/15801),
the expectation is that the increase in MaxEB \"will slightly increase
the time it takes to calculate the next proposer index (lower values of
EB will result in lower probability of selection, thus more iterations
of the loop)\". We demonstrate this in the graph below where we use a
negative binomial to estimate the number of failed proposer eligibility
checks before a solo validator passes the check.

We only looked at proposers with 32 ETH effective balances, so the
probability of passing the proposer check was 0.016. Therefore the graph
shows us the probability distribution of the number of failures of the
check before the first successful proposer check \[\^2\].

When we iterate through the shuffled validator indices, we observe that
before a \"successful\" candidate is reached, all the candidates ahead
of the eventual proposer in the shuffled index had to have been
rejected. However, given the large active validator set, the probability
calculations based on a large finite sample assumption hold and do not
materially change the calculations, and are valid in this case.

\![proposer_negbinomial.png\](upload://sci7OqCaOni1B4a3RInijgnbyAN.png)
\*\<center\>Figure 3: Visual representation of the number of failures
for solo stakers before a candidate single validator passes the proposer
check\</center\>\*\<br\>

The $median$ value for the number of failures is 43, i.e. we can expect
that half of the time more than 43 iterations will be required and half
the time fewer than 43 iterations.

Apart from the median, it is also interesting to quantify other
probabilities, such as: 1. Probability of fewer than 100 iterations $=$
0.7962 2. Probability of more than 100 iterations $=$ 1 - 0.7962 $=$
0.2038 3. Probability of more than 200 iterations $=$ 0.0422 4.
Probability of more than 300 iterations $=$ 0.0087 5. Probability of
more than 400 iterations $=$ 0.0018

\## Staker dilemma: Consolidate or not?? The probability of being
selected as the candidate from the shuffled consolidated validator set
of size \*n\* is the same for each validator, regardless of the extent
of consolidation, viz. $\frac{1}{n}$.

So if staker \*A\* has 64 single validators and staker \*B\* has one
consolidated staker, then the probability that a validator from staker
\*A\* or staker \*B\* is the next proposer is calculated as follows:

$P(staker \texttt{ } A \texttt{ } is \texttt{ } next \texttt{ } proposer) =
 \frac{64}{n} * \frac{32}{2048} = \frac{1}{n}$ \<br\>
$P(staker \texttt{ } B \texttt{ } is \texttt{ } next \texttt{ } proposer) =
 \frac{1}{n} * 1 = \frac{1}{n}$

From this we can deduce that for this scenario, as far as proposer
selection is concerned, it is equally likely for for a large staker that
one of their validators will be the next proposer, regardless of whether
they decide to consolidate validators to the full MaxEB, or leave them
all as unconsolidated.

However, there are several other considerations for a staker to take
into account when deciding on a consolidation strategy, such as rewards
earned, slashing risk and penalties that vary with effective balance.
Proposer selection is just one part of the puzzle.

\## References \<a name=\"first\"\>\[1\]\</a\> Viet Tung Hoang, Ben
Morris & Phillip Rogaway (2014): An Enciphering Scheme Based on a Card
Shuffle. \[arXiv:1208.1176\](https://arxiv.org/abs/1208.1176) \<br\> \<a
name=\"second\"\>\[2\]\</a\>: S. Johnson et al. (2013): "Modeling the
viability of the free-ranging cheetah population in Namibia: an
object-oriented Bayesian network approach," Ecosphere, vol. 4, no. 7,
\[link to journal
article\](https://esajournals.onlinelibrary.wiley.com/doi/pdf/10.1890/ES12-00357.1)\[EIP-7251\](https://eips.ethereum.org/EIPS/eip-7251)
proposes increasing the \*MAX_EFFECTIVE_BALANCE\* constant from 32 ETH
to 2,048 ETH, but the \*minimum effective balance\* required to join as
a validator remains unchanged at 32 ETH.

The expectation is that \"Proposer selection is already weighted by the
ratio of their effective balance to MAX_EFFECTIVE_BALANCE. Due to the
lower probabilities, this change will slightly increase the time it
takes to calculate the next proposer index.\"

We undertook some additional analysis of proposer selection which is
summarised in this blog post.

\## Proposer selection process

Proposer selection is a two-stage process: 1. Being \*selected as the
candidate\* from the list of shuffled validator indices. 2. Passing the
\*proposer eligibility\* check

\## The swap-or-not-shuffle technique \[\[1\]\](#first) is used to
shuffle the validator indices in preparation for the selection of a
block proposer.

This is done in
\[compute_shuffled_index\](https://github.com/ethereum/consensus-specs/blob/9c35b7384e78da643f51f9936c578da7d04db698/specs/phase0/beacon-chain.md#compute_shuffled_index).
The computation to determine the proposer for the next block is done in
\[compute_proposer_index\](https://github.com/ethereum/consensus-specs/blob/9c35b7384e78da643f51f9936c578da7d04db698/specs/phase0/beacon-chain.md#compute_proposer_index)
(shown below):

    python def compute_proposer_index(state: BeaconState, indices:
Sequence\[ValidatorIndex\], seed: Bytes32) -\> ValidatorIndex: \"\"\"
Return from ''indices'' a random index sampled by effective balance.
\"\"\" assert len(indices) \> 0 MAX_RANDOM_BYTE = 2\*\*8 - 1 i =
uint64(0) total = uint64(len(indices)) while True: candidate_index =
indices\[compute_shuffled_index(i random_byte = hash(seed +
uint_to_bytes(uint64(i // 32)))\[i effective_balance =
state.validators\[candidate_index\].effective_balance if
effective_balance \* MAX_RANDOM_BYTE \>= MAX_EFFECTIVE_BALANCE \*
random_byte: return candidate_index i += 1    

Therefore, we iterate through the shuffled indices, starting with the
first entry and then check whether it passes the selection criteria. If
it doesn't, then the next validator index in the array goes through the
same check.

As we can see from the code, the validator's effective balance (EB) is
multiplied by 255 (i.e. $MAX\_RANDOM\_BYTE = 2^8 - 1 = 255$) and then
compared to the product of the generated $random\_byte$ $(r)$ and
$MAX\_EFFECTIVE\_BALANCE = 2,048 ETH$.

Figure 1 (a) is an exposition of random byte values generated for
716,800 validators from the \*random_byte\* assignment statement below.
Superimposed on the histogram of these random byte integers is a uniform
distribution. As expected, the random bytes appear to visually resemble
values drawn from a uniform distribution: $r \sim U(0,255)$. We confirm
this in the Q--Q plot in Figure 1 (b). Hence we can assume in subsequent
calculations that the random bytes, $r$, have a uniform distribution,
$r \sim U(0,255)$.

    python random_byte = hash(seed + uint_to_bytes(uint64(i // 32)))\[i
    \<img src=\"upload://qqt3TqttDP5dBFg8SjSsUOQlGt0.png\" width=\"340\"
height=\"355\"\>\<img src=\"upload://uSaQN1GdZw73WfZAuRrvqOfG1HS.png\"
width=\"340\" height=\"340\"\> \<br\>
&nbsp;&nbsp;&nbsp;&nbsp;&nbsp;&nbsp;&nbsp;&nbsp;&nbsp;&nbsp;&nbsp;&nbsp;&nbsp;&nbsp;&nbsp;&nbsp;&nbsp;&nbsp;&nbsp;&nbsp;&nbsp;&nbsp;&nbsp;&nbsp;&nbsp;&nbsp;&nbsp;&nbsp;&nbsp;&nbsp;&nbsp;&nbsp;&nbsp;&nbsp;&nbsp;&nbsp;&nbsp;&nbsp;&nbsp;&nbsp;&nbsp;&nbsp;&nbsp;&nbsp;&nbsp;&nbsp;&nbsp;&nbsp;&nbsp;&nbsp;&nbsp;&nbsp;(a)&nbsp;&nbsp;&nbsp;&nbsp;&nbsp;&nbsp;&nbsp;&nbsp;&nbsp;&nbsp;&nbsp;&nbsp;
&nbsp;&nbsp;&nbsp;&nbsp;&nbsp;&nbsp;&nbsp;&nbsp;&nbsp;&nbsp;&nbsp;&nbsp;&nbsp;&nbsp;
&nbsp;&nbsp;&nbsp;&nbsp;&nbsp;&nbsp;&nbsp;&nbsp;&nbsp;&nbsp;&nbsp;&nbsp;&nbsp;&nbsp;
&nbsp;&nbsp;&nbsp;&nbsp;&nbsp;&nbsp;&nbsp;&nbsp;&nbsp;&nbsp;&nbsp;&nbsp;&nbsp;&nbsp;
&nbsp;&nbsp;&nbsp;&nbsp;&nbsp;&nbsp;&nbsp;&nbsp;&nbsp;&nbsp;&nbsp;&nbsp;&nbsp;&nbsp;(b)

\*Figure 1: (a) Distribution of 716,800 random bytes generated from the
spec (b) Quantile-quantile plot of the generated random bytes against a
uniform distribution - U(0,255)\*

The probability of a validator being the proposer if their index was
selected from the list is calculated as follows (where $\therefore$
means \*therefore\*):

$P(proposer \texttt{ } check \texttt{ }passed) = P(EB * 255 \geqslant MaxEB * r)$,
where $r \sim U(0,255)$, $r = random\_byte$,
$EB = validator\_e$\*ff\*$ective\_balance$\<br\>
$\therefore P(proposer \texttt{ } check \texttt{ }passed) = P\left( r \leqslant  \frac{255*EB}{MaxEB}\right)$
\<br\>
$\therefore \textit{if } EB = MaxEB \implies P(proposer \texttt{ } check \texttt{ }passed) = 1$

In other words, if the effective balance of the candidate validator
equalled the maximum effective balance, then the validator becomes the
proposer with probability 1.

---

When the maximum effective balance is increased to 2,048 ETH, the
probability of passing the proposer eligibility test will vary depending
on the extent of validator consolidation.

As before, a fully consolidated validator with an EB of \*\*2, 048 ETH
(64 \* 32 ETH)\*\* will become a proposer with a probability of 1 if the
validator's index was randomly selected as the next candidate.

However, if the randomly selected candidate validator has an EB of:

\*\*32 ETH\*\*:
$P(proposer \texttt{ } check \texttt{ }passed) = P\left(r \leqslant \frac{255*32}{2048} \right)$\<br\>
$\therefore P(proposer \texttt{ } check \texttt{ }passed) = P(r \leqslant 3.98) = \left(\frac{3.98-0}{255}\right) = 0.016$
\<br\>
$\therefore P(proposer \texttt{ } check \texttt{ }passed) \equiv \left(\frac{32}{2048}\right) = 0.016$
\<br\> \*\*64 ETH (2 \* 32 ETH)\*\*
$\therefore P(proposer \texttt{ } check \texttt{ } passed) = \left(\frac{64}{2048}\right) = 0.031$
\<br\> \*\*160 ETH (5 \* 32 ETH)\*\*
$\therefore P(proposer \texttt{ } check \texttt{ } passed) = \left(\frac{160}{2048}\right) = 0.078$
\<br\> \*\*320 ETH (10 \* 32 ETH)\*\*
$\therefore P(proposer \texttt{ } check \texttt{ }passed) = \left(\frac{320}{2048}\right) = 0.156$\<br\>
\*\*960 ETH (30 \* 32 ETH)\*\*
$\therefore P(proposer \texttt{ } check \texttt{ }passed) = \left(\frac{960}{2048}\right) = 0.469$

Figure 2 below depicts the increase in passing the proposer eligibility
test as the validator's effective balance increases.

&nbsp;&nbsp;&nbsp;&nbsp;&nbsp;&nbsp;&nbsp;&nbsp;&nbsp;&nbsp;&nbsp;&nbsp;&nbsp;&nbsp;&nbsp;&nbsp;&nbsp;&nbsp;&nbsp;&nbsp;&nbsp;&nbsp;&nbsp;&nbsp;&nbsp;&nbsp;&nbsp;&nbsp;&nbsp;&nbsp;&nbsp;&nbsp;&nbsp;&nbsp;\<img
src=\"upload://iHtiVZv1APGBghGV1Nia16WZJaT.png\" width=\"450\"
height=\"380\>̈ \*Figure 2: Probability of passing the proposer
eligibility check for a candidate validator with an EB ranging from 32
to 2,048 ETH\*\<br\> In other words, if a validator's index is selected
as the next candidate, the probability of passing the selection check to
propose the next block varies from 0.016 for an unconsolidated validator
(32ETH) to 1 for a fully consolidated validator (2,048ETH).

-----

Assuming a validator set size of 716,800, then the probability of any
validator, regardless of their effective balance, being first in the
list of shuffled validator indices (i.e. the candidate to be assessed as
the next block proposer) is: \<br\>
$P(candidate) = \frac{1}{(Active \texttt{ } validator \texttt{ } set \texttt{ } size)} = \frac{1}{ 716,800} = 1.395*10^{-6}$.

Currently (i.e. prior to EIP-7521), providing a validator maintains its
effective balance at 32 ETH, once its index is selected as the next
candidate to be checked, it would pass the proposer selection test with
certainty (i.e. probability of $1$, or $100\%$).

Putting it another way:\<br\> \*Given EB=32 ETH\*, then the probability
of being selected as the $1^{st}$ candidate and becoming the next
proposer is calculated as follows:\<br\> $P(candidate \texttt{ }$ $\&$
$\texttt{ }proposer$ $check$ $passed)$ \<br\>
$= P(candidate) * P(proposer$ $check$ $passed / candidate)$ \<br\>
$= P(candidate) * P(proposer$ $check$ $passed)$ &nbsp; &nbsp; &nbsp;
&nbsp; (since these two events are independent) \<br\>
$= \left(\frac{1}{716800}\right) * 1 \approx 1.395 * 10^{-6}$

After MaxEB = 2,048 ETH, this changes to:\<br\> $P(candidate$ $\&$
$proposer$ $check$ $passed) =$
$\left(\frac{1}{716800}\right) * \left(\frac{32}{2048}\right)$
$\approx 2.18*10^{-8}$

For the current MaxEB of 32 ETH, we know that if the validator is
selected as the $1^{st}$ candidate in the shuffled index, then they will
definitely be the proposer for the next block.

When we consider the increased MaxEB proposal, then it may be quite
likely that a validator is selected as the next candidate validator for
the proposer eligibility test at subsequent rounds (e.g.
$2^{nd}, 3^{rd}, ..., n^{th}$), because a selected candidate will be
less likely to pass the proposer check.

Therefore we calculate the probability that a 32 ETH validator, $v$, is
selected at round $i$ given that all the previous rounds
$(1,2, ... (i-1))$ were unsuccessful, i.e. we sum over all the possible
rounds that this validator may have been selected.

For example, if validator $v$ is selected at round 3, there would have
been a rejection of another validator at round 1 and round 2. Stating
this more generally: \<br\>
$P(validator \texttt{ } v \texttt{ } selected \texttt{ } as \texttt{ } next \texttt{ } proposer) = \sum_{i=1}^{716,800} p_i * \left( \prod_{j=1}^i (1-p_{j-1}) \right), \texttt{ } where$
\<br\>
$p_i = P(round \texttt{ }i \texttt{ } proposer) = P(round \texttt{ } i \texttt{ } candidate) * P(passing \texttt{ } proposer \texttt{ } eligibility)$,
$and 
\texttt{ } p_0 = 0$ \<br\> This probability is also $1.395 * 10^{-6}$.

In summary, with an increased MaxEB, a solo staker would be selected as
the next proposer with a probability of $2.18*10^{-8}$, from the current
$1.395 * 10^{-6}$ probability if they are the \*\*first\*\* candidate in
the shuffled index. However, their probability over all the possible
outcomes will be the same as is currently the case, if all the other
validators have the same EB and hence the same probability of passing
the proposer check.

It gets a bit more complex when the minimal solo staker is competing
against validators that are more likely to pass the proposer eligibility
check, e.g. after implementation of EIP-7521.

The probability calculations above assumes a validator set with no
consolidation, i.e. the validator set size remained unchanged with the
introduction of EIP-7521. In practice this is highly unlikely.
Therefore, we consider an example scenario.

\### Example Scenario: \*Active validator set has validators with
varying EBs (i.e. stakers may choose different strategies for validator
consolidation)\* To the best of our knowledge, the analyses to date have
mainly been for a homogeneous validator set, i.e. all validators have
the same effective balance, either unconsolidated (32 ETH) or fully
consolidated (2,048 ETH). This includes the analysis we conducted above.

Hence, a more interesting scenario would be an active validator set with
a variety of effective balances.

Therefore, let us assume that we have an active validator set with
716,800 validators prior to EIP-7251. The stakers in this validator set
choose to combine some of their validators to a greater or lesser
extent, and may in fact leave several of their validators
unconsolidated, i.e. at 32 ETH.

We visualise this hypothetical consolidation of the validator set in
Figure 3 below.

We group the stakers into five categories: - Small-scale solo stakers
(32 - a few hundred ETH) - Large-scale individual solo stakers (1000+
ETH) - Large-scale institutional solo stakers (ie. companies staking
their own ETH) - Centralised staking pools - Semi-decentralised staking
pools (e.g. Rocketpool, Lido\... )

\![\](upload://lNZenxS8kjiAuO8A48qvOw8JoDg.png)
&nbsp;&nbsp;&nbsp;&nbsp;&nbsp;&nbsp;&nbsp;&nbsp;&nbsp;&nbsp;&nbsp;&nbsp;&nbsp;&nbsp;&nbsp;&nbsp;&nbsp;&nbsp;\*Figure
3: Visual representation of the example scenario for validator
consolidation\* \<br\>

We put a potential distribution over the various validator
consolidations in our scenario for illustrative purposes. The
distributions can be adjusted as required.

The \*Validator numbers after consolidation\* row (green) shows the
reduced validator numbers for each category, using the consolidations
shown in each column (yellow).

Based on the chosen configuration, the total validator set size reduces
to 329,810 (last value in the diagram).

\##

\### Bayesian network for proposer selection We built a simple Bayesian
network (BN) to illustrate the dependencies between the different nodes
in the BN model for the example scenario described above (Figures 4 and
5).

A Bayesian network (BN) is an acyclic directed graph of nodes and edges.
The nodes represent the key factors of the system or problem being
modelled, and the edges between the nodes indicate dependencies.
Uncertainty and the stength of the dependencies between connected nodes
are explicitly captured in the node probability tables that are attached
to each node in the model. An object-oriented version of a BN (OOBN) may
be used to make BNs less cluttered and more readible, by grouping
related nodes and processes in OOBN submodels. \[\[2\]\](#second)

\<img src=\"upload://uUDZCAzVzYTp8fdEWGhKPAvh4QL.png\" width=\"235\"
height=\"300\"\>&nbsp; &nbsp; \<img
src=\"upload://eHb9l4uK27PcsufEVDIULFj59hM.png\" width=\"430\"\> \<br\>
\*Figure 4: OOBN for proposer
selection\*&nbsp;&nbsp;&nbsp;&nbsp;&nbsp;&nbsp;&nbsp;&nbsp;&nbsp;&nbsp;&nbsp;&nbsp;&nbsp;&nbsp;&nbsp;&nbsp;&nbsp;&nbsp;&nbsp;&nbsp;
\*Figure 5: OOBN subnet for probability calculations\*

We calculate the probability that a validator is selected and passes the
check for proposer eligibility, as the product of the probability of
being the candidate index and the probability of passing the proposer
check, since these two events are independent, i.e.

$P(A \cap B) = P(A/B)P(B) = P(A)P(B)$.

The probabilities shown in Figure 2 above are used in the node
probability table for BN node \*Proposer check\*, which captures the
probability that a validator will pass the proposer eligibility check
for the various consolidated validator sizes.

Using the logic from the equation above and the previous calculations
for the probability that a validator passes the eligibility check, we
can now determine: - the probability for a validator from each
consolidation group (e.g. single, 2-fold, etc) to be selected as a
candidate validator from the shuffled validator indices, and - the
probability that this candidate type will pass the eligibility check for
being a proposer.

These probabilities are shown in Table 1 below in the last two columns,
and are used to populate the probability node tables for BN nodes
\*Validator type selected as candidate for proposer duty\* and
\*Proposer for next block\*, respectively.

For the example scenario described above and assuming a validator set
size of 716,800, we can use the BN model to gain some insights into the
probabilities of different 'types' of validators. By 'type' we mean the
extent to which the validator has been consolidated.

Running this model generates the marginal probabilities shown in Figure
6(a). The proportion of validator categories in the reduced validator
set are visible in BN node \*Consolidated validator types\* and are in
the second column of Table 1.

We see that based on the configuration of the various types of
validators, the active validator set reduced to 329,810.

\### Table 1: Probability that validator type is selected as a candidate
proposer \| \*\*Type of\*\* \<br\>
\*\*validator\*\*\<br\>\*\*consolidation\*\* \| \*\*Proportion\*\*
\<br\>\*\*of validators\*\* \<br\> \*\*consolidating\*\* \|\*\*Number
of\*\* \<br\> \*\*validators in\*\* \<br\> \*\*consolidated\*\* \<br\>
\*\*validator set\*\*\|\*\*P(validator type\*\* \<br\> \*\*selected
as\*\* \<br\> \*\*candidate)\*\* \|\*\*P(selected
validator\*\*\<br\>\*\*type is next\*\* \<br\> \*\*proposer)\*\* \|
\|:----------------:\|:----------------:\|:------------:\|:----------------:\|:----------------:\|
\| Single\| 28.75 \| Partial (2-fold)\| 25.75 \| Partial (5-fold)\|
15.00 \| Partial (10-fold)\| 9.00 \| Partial (30-fold)\| 8.50 \| Full
(64-fold)\| 13.00 \|\*\*TOTAL\*\*\| \*\*100.00

\<img src=\"upload://12f7eIF4DLSw9dK1nyOQdIDBjX5.png\" width=\"200\"
height=\"300\"\>&nbsp;&nbsp;&nbsp;&nbsp;&nbsp;&nbsp;&nbsp;&nbsp;&nbsp;&nbsp;\<img
src=\"upload://wrYcCqC3ZsqnJK5DnBgMa7YouiT.png\" width=\"200\"
height=\"300\"\>&nbsp;&nbsp;&nbsp;&nbsp;&nbsp;&nbsp;&nbsp;&nbsp;&nbsp;&nbsp;\<img
src=\"upload://xAOjVBPK0ha7wc8oUmSpdOtgRxP.png\" width=\"200\"
height=\"300\"\> \<br\>
&nbsp;&nbsp;&nbsp;&nbsp;&nbsp;&nbsp;&nbsp;&nbsp;&nbsp;&nbsp;&nbsp;&nbsp;&nbsp;&nbsp;&nbsp;&nbsp;&nbsp;&nbsp;&nbsp;&nbsp;&nbsp;&nbsp;&nbsp;&nbsp;&nbsp;&nbsp;&nbsp;&nbsp;&nbsp;&nbsp;&nbsp;&nbsp;(a)
&nbsp;&nbsp;&nbsp;&nbsp;&nbsp;&nbsp;&nbsp;&nbsp;&nbsp;&nbsp;&nbsp;&nbsp;
&nbsp;&nbsp;&nbsp;&nbsp;&nbsp;&nbsp;&nbsp;&nbsp;&nbsp;&nbsp;&nbsp;&nbsp;&nbsp;&nbsp;
&nbsp;&nbsp;&nbsp;&nbsp;&nbsp;&nbsp;&nbsp;&nbsp;&nbsp;&nbsp;&nbsp;&nbsp;&nbsp;&nbsp;
&nbsp;&nbsp;&nbsp;&nbsp;&nbsp;&nbsp;&nbsp;(b)&nbsp;&nbsp;&nbsp;&nbsp;&nbsp;&nbsp;&nbsp;&nbsp;&nbsp;&nbsp;&nbsp;&nbsp;&nbsp;&nbsp;
&nbsp;&nbsp;&nbsp;&nbsp;&nbsp;&nbsp;&nbsp;&nbsp;&nbsp;&nbsp;&nbsp;&nbsp;&nbsp;&nbsp;
&nbsp;&nbsp;&nbsp;&nbsp;&nbsp;&nbsp;&nbsp;&nbsp;&nbsp;&nbsp;&nbsp;&nbsp;&nbsp;&nbsp;
&nbsp;&nbsp;&nbsp;&nbsp;&nbsp;&nbsp;&nbsp;(c) \<br\>

\<img src=\"upload://mY48b4dkvt83EO4A1Jq5WHR2r0z.png\" width=\"200\"
height=\"300\"\>&nbsp;&nbsp;&nbsp;&nbsp;&nbsp;&nbsp;&nbsp;&nbsp;&nbsp;&nbsp;\<img
src=\"upload://5LRzB4k9Rxt81gegHerxjopptK2.png\" width=\"200\"
height=\"300\"\>&nbsp;&nbsp;&nbsp;&nbsp;&nbsp;&nbsp;&nbsp;&nbsp;&nbsp;&nbsp;\<img
src=\"upload://nBR4g3Nd6ya1pLJ3cIpVgcqTZxs.png\" width=\"200\"
height=\"300\>̈

\<br\>&nbsp;&nbsp;&nbsp;&nbsp;&nbsp;&nbsp;&nbsp;&nbsp;&nbsp;&nbsp;&nbsp;&nbsp;&nbsp;&nbsp;&nbsp;&nbsp;&nbsp;&nbsp;&nbsp;&nbsp;&nbsp;&nbsp;&nbsp;&nbsp;&nbsp;&nbsp;&nbsp;&nbsp;&nbsp;&nbsp;&nbsp;&nbsp;(d)
&nbsp;&nbsp;&nbsp;&nbsp;&nbsp;&nbsp;&nbsp;&nbsp;&nbsp;&nbsp;&nbsp;&nbsp;
&nbsp;&nbsp;&nbsp;&nbsp;&nbsp;&nbsp;&nbsp;&nbsp;&nbsp;&nbsp;&nbsp;&nbsp;&nbsp;&nbsp;
&nbsp;&nbsp;&nbsp;&nbsp;&nbsp;&nbsp;&nbsp;&nbsp;&nbsp;&nbsp;&nbsp;&nbsp;&nbsp;&nbsp;
&nbsp;&nbsp;&nbsp;&nbsp;&nbsp;&nbsp;&nbsp;(e)&nbsp;&nbsp;&nbsp;&nbsp;&nbsp;&nbsp;&nbsp;&nbsp;&nbsp;&nbsp;&nbsp;&nbsp;&nbsp;&nbsp;
&nbsp;&nbsp;&nbsp;&nbsp;&nbsp;&nbsp;&nbsp;&nbsp;&nbsp;&nbsp;&nbsp;&nbsp;&nbsp;&nbsp;
&nbsp;&nbsp;&nbsp;&nbsp;&nbsp;&nbsp;&nbsp;&nbsp;&nbsp;&nbsp;&nbsp;&nbsp;&nbsp;&nbsp;
&nbsp;&nbsp;&nbsp;&nbsp;&nbsp;&nbsp;&nbsp;(f)

\*Figure 6: Running proposer selection model for various scenarios\*

--- \### Running the OOBN model \#### Scenario 1: Running the model with
no evidence provided (a) The result of running the BN model is shown in
Figure 6(a). Therefore, assuming the active validator set consists of
the staking categories as described along with their respective example
consolidation strategies, the validator set will be reduced to 329,810
from 716,800. This adjusted validator set has different proportions of
consolidated validators as shown in node \*Consolidated validator
types\*. Across this validator set, the proportion of validators that
will pass the proposer test is 20.8

The probability of being the next proposer depends on being selected as
the next candidate and then passing the test. It appears
counter-intuitive that the probability is so much smaller than the
individual probabilities in the BN. The reason for this becomes clearer
when we look at the conditional probability table of this node:
\![NextProposerCPT\](upload://f6uyuaUWgf4396dOmytLrTa6ozi.png)

\#### Scenario 2: Running the model with all validators having 32 ETH
(b) If we enter evidence (shown as a state of a node being red) that the
validator set consists entirely of unconsolidated validators, and
therefore there is no reduction in the size of the validator set, the
probability of selecting an unconsolidated validator type is
understandably 100

\#### Scenario 3: Running the model with all validators having 64 ETH
(\*\*$\textbf{c}$\*\*) In this scenario we assume that all stakers
decided to consolidate their validators into 64 ETH validators. Here the
validator set size reduces to 358,400. As in Scenario 2, the first
selected validator type will be a validator with 64 ETH, i.e. 100

\#### Scenario 4: Running the model with all validators having MaxEB
(2,048 ETH) (d) In this scenario we assume that all stakers decided to
consolidate their validators to the maximum allowed, i.e. 2,048 ETH.
This is the largest reduction in validator set size, being just 11,200.
As in Scenarios 2 & 3, the first selected validator type will definitely
be a validator with an EB of MaxEB. For MaxEB the probability of passing
the proposer check is 3.12

\#### Scenario 5: Running the model assuming the validator set consists
only of small-scale stakers applying the example consolidation strategy
(e) The small staker group is assumed to mainly consist of validators
with 32 or 64 ETH, with a small proportion consolidated into 160 ETH
validators. Based on this strategy, the validator set reduces to
458,752. Across these small staker validators, the proportion that will
pass the proposer test is 3.44 38.75

\#### Scenario 6: assuming the validator set consists only of
semi-decentralized staking pools applying the example consolidation
strategy (f) In the example consolidation strategy for
semi-decentralized staking pools, they are assumed to have a fairly even
spread across the various extents of consolidation, with a slight
majority of single stake validators. Based on this strategy, the
validator set reduces to 312,853. Across this validator set, the
proportion of validators that will pass the proposer test is 28.13 26.11

---

\### Online version of Proposer Selection OOBN model We used \[HUGIN
EXPERT\](https://www.hugin.com/) when constructing the proposer
selection model. Hugin made the proposer selection OOBN model freely
available \[online\](https://demo.hugin.com/example/ProposerSelection)
with some widgets to allow interaction with the model.

\##

\## Increased proposer selection time As pointed out in the ethresear.ch
post \[Increase the MAX_EFFECTIVE_BALANCE -- a modest
proposal\](https://ethresear.ch/t/increase-the-max-effective-balance-a-modest-proposal/15801),
the expectation is that the increase in MaxEB \"will slightly increase
the time it takes to calculate the next proposer index (lower values of
EB will result in lower probability of selection, thus more iterations
of the loop)\". We demonstrate this in the graph below where we use a
negative binomial to estimate the number of failed proposer eligibility
checks before a solo validator passes the check.

We only looked at proposers with 32 ETH effective balances, so the
probability of passing the proposer check was 0.016. Therefore the graph
shows us the probability distribution of the number of failures of the
check before the first successful proposer check \[\^2\].

When we iterate through the shuffled validator indices, we observe that
before a \"successful\" candidate is reached, all the candidates ahead
of the eventual proposer in the shuffled index had to have been
rejected. However, given the large active validator set, the probability
calculations based on a large finite sample assumption hold and do not
materially change the calculations, and are valid in this case.

\![proposer_negbinomial.png\](upload://sci7OqCaOni1B4a3RInijgnbyAN.png)
\*\<center\>Figure 3: Visual representation of the number of failures
for solo stakers before a candidate single validator passes the proposer
check\</center\>\*\<br\>

The $median$ value for the number of failures is 43, i.e. we can expect
that half of the time more than 43 iterations will be required and half
the time fewer than 43 iterations.

Apart from the median, it is also interesting to quantify other
probabilities, such as: 1. Probability of fewer than 100 iterations $=$
0.7962 2. Probability of more than 100 iterations $=$ 1 - 0.7962 $=$
0.2038 3. Probability of more than 200 iterations $=$ 0.0422 4.
Probability of more than 300 iterations $=$ 0.0087 5. Probability of
more than 400 iterations $=$ 0.0018

\## Staker dilemma: Consolidate or not?? The probability of being
selected as the candidate from the shuffled consolidated validator set
of size \*n\* is the same for each validator, regardless of the extent
of consolidation, viz. $\frac{1}{n}$.

So if staker \*A\* has 64 single validators and staker \*B\* has one
consolidated staker, then the probability that a validator from staker
\*A\* or staker \*B\* is the next proposer is calculated as follows:

$P(staker \texttt{ } A \texttt{ } is \texttt{ } next \texttt{ } proposer) =
 \frac{64}{n} * \frac{32}{2048} = \frac{1}{n}$ \<br\>
$P(staker \texttt{ } B \texttt{ } is \texttt{ } next \texttt{ } proposer) =
 \frac{1}{n} * 1 = \frac{1}{n}$

From this we can deduce that for this scenario, as far as proposer
selection is concerned, it is equally likely for for a large staker that
one of their validators will be the next proposer, regardless of whether
they decide to consolidate validators to the full MaxEB, or leave them
all as unconsolidated.

However, there are several other considerations for a staker to take
into account when deciding on a consolidation strategy, such as rewards
earned, slashing risk and penalties that vary with effective balance.
Proposer selection is just one part of the puzzle.

\## References \<a name=\"first\"\>\[1\]\</a\> Viet Tung Hoang, Ben
Morris & Phillip Rogaway (2014): An Enciphering Scheme Based on a Card
Shuffle. \[arXiv:1208.1176\](https://arxiv.org/abs/1208.1176) \<br\> \<a
name=\"second\"\>\[2\]\</a\>: S. Johnson et al. (2013): "Modeling the
viability of the free-ranging cheetah population in Namibia: an
object-oriented Bayesian network approach," Ecosphere, vol. 4, no. 7,
\[link to journal
article\](https://esajournals.onlinelibrary.wiley.com/doi/pdf/10.1890/ES12-00357.1)
