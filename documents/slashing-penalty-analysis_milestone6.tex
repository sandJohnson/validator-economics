\documentclass[UTF8]{article}
%\usepackage{natbib}
\usepackage{geometry}                % See geometry.pdf to learn the layout options. There are lots.
\geometry{a4paper}                   % ... or a4paper or a5paper or ... 

%\geometry{landscape}                % Activate for for rotated page geometry
\usepackage{breakurl}     
\usepackage{graphicx}
\usepackage{graphbox}
\usepackage{hyperref}
\usepackage{amssymb}
\usepackage{epstopdf}
\usepackage{algorithm}
\usepackage{amsmath}
\usepackage[noend]{algpseudocode}
\usepackage{enumitem}
\usepackage{multirow}
\usepackage{hhline}
\usepackage[acronym]{glossaries}
\usepackage[hang,footnotesize,bf]{caption}
\DeclareGraphicsRule{.tif}{png}{.png}{`convert #1 `dirname #1`/`basename #1 .tif`.png}

% Copied from StackOverflow for code snippets 
% ------------------------------------------------------------
\usepackage{listings}
\usepackage{color}

\definecolor{dkgreen}{rgb}{0,0.6,0}
\definecolor{gray}{rgb}{0.5,0.5,0.5}
\definecolor{mauve}{rgb}{0.58,0,0.82}

\lstset{frame=tb,
  language=Python,
  aboveskip=3mm,
  belowskip=3mm,
  showstringspaces=false,
  columns=flexible,
  basicstyle={\small\ttfamily},
  numbers=none,
  numberstyle=\tiny\color{gray},
  keywordstyle=\color{blue},
  commentstyle=\color{dkgreen},
  stringstyle=\color{mauve},
  breaklines=true,
  breakatwhitespace=true,
  tabsize=3
}


% Copied from Stack Exchange for ToDo items
% -----------------------------------------------------------
\usepackage{xargs} 
\usepackage[pdftex,dvipsnames]{xcolor}  % Coloured text etc.
% 
\usepackage[colorinlistoftodos,prependcaption,textsize=tiny]{todonotes}
\newcommandx{\unsure}[2][1=]{\todo[linecolor=red,backgroundcolor=red!25,bordercolor=red,#1]{#2}}
\newcommandx{\change}[2][1=]{\todo[linecolor=blue,backgroundcolor=blue!25,bordercolor=blue,#1]{#2}}
\newcommandx{\info}[2][1=]{\todo[linecolor=OliveGreen,backgroundcolor=OliveGreen!25,bordercolor=OliveGreen,#1]{#2}}
\newcommandx{\improvement}[2][1=]{\todo[linecolor=Plum,backgroundcolor=Plum!25,bordercolor=Plum,#1]{#2}}
\newcommandx{\feedback}[2][1=]{\todo[linecolor=Goldenrod,backgroundcolor=Goldenrod!25,bordercolor=Coldenrod,#1]{#2}}
\newcommandx{\thiswillnotshow}[2][1=]{\todo[disable,#1]{#2}}


\title{Validator Economics: Variable min validator deposit size\\
\vspace{4pt}
\large EF Academic Grant ID: FY23-1030\\
\vspace{16pt}
SLASHING \& PENALTIES - MILESTONE 6 }
\vspace{16pt}
\author{Sandra Johnson\\
Consensys Software R\&D}
\date{\today}                                           % Activate to display a given date or no date

\begin{document}
\maketitle


% Acronym definitions
\newacronym{apr}{APR}{annual percentage rate}
\newacronym{bls}{BLS}{Boneh–Lynn–Shacham}
\newacronym{bn}{BN}{Bayesian network}
\newacronym{cbeth}{cbETH}{Coinbase wrapped staked ETH}
\newacronym{cdf}{CDF}{cumulative distribution function}
\newacronym{cl}{CL}{consensus layer}
\newacronym{cpt}{CPT}{conditional probability table}
\newacronym{dos}{DoS}{denial of service}
\newacronym{dvt}{DVT}{distributed validator technology}
\newacronym{eb}{EB}{effective balance}
\newacronym{ef}{EF}{Ethereum Foundation}
\newacronym{eip}{EIP}{Ethereum Improvement Proposal}
\newacronym{el}{EL}{execution layer}
\newacronym{epbs}{ePBS}{enshrined PBS}
\newacronym{ffg}{FFG}{Friendly finality gadget}
\newacronym{fxs}{FXS}{Frax share}
\newacronym{ghost}{GHOST}{Greedy Heaviest-Observed Sub-Tree}
\newacronym{ldo}{LDO}{Lido DAO}
\newacronym{lmd}{LMD}{Latest message driven}
\newacronym{mev}{MEV}{maximal extractable value}
\newacronym{mm}{MM}{MetaMask}
\newacronym{npt}{NPT}{node probability table}
\newacronym{p2p}{p2p}{peer-to-peer}
\newacronym{pbs}{PBS}{proposer builder separation}
\newacronym{pc}{PC}{personal computer}
\newacronym{pdf}{PDF}{probability density function}
\newacronym{pos}{PoS}{proof of stake}
\newacronym{pr}{PR}{pull request}
\newacronym{rig}{RIG}{Robust Incentives Group}
\newacronym{rpl}{RPL}{Rocket Pool}
\newacronym{ssf}{SSF}{single-slot finality}
\newacronym{steth}{stETH}{Lido staked Ether}
\newacronym{ups}{UPS}{uninterruptable power supply}
\newacronym{vrf}{VRF}{verifiable random function}

% ------------------------------------------------
\section{Slashing penalty analysis}
% ------------------------------------------------
The main motivation for this
\href{https://notes.ethereum.org/@mikeneuder/slashings-eip-7251}{post} is the
concern of large stakers regarding slashing penalties for consolidated
validators. The proposal consists of two parts:
\begin{enumerate}
  \item Changes to existing penalties:
    \begin{itemize}
      \item Changing the \textit{initial penalty} to be either fixed, or scaled
        sublinearly
      \item Changing the \textit{correlation penalty} to scale quadratically
        rather than linearly.
    \end{itemize}
  \item Unchanged penalties:
    \begin{itemize}
      \item \textit{Attestation penalties}
      \item \textit{Inactivity leak penalties}
    \end{itemize}
\end{enumerate}

% ---------------------------------------------------------------
\subsection{Validator slashing and penalties}
% ---------------------------------------------------------------
\label{slashing} \subsubsection*{Slashing} The situations that lead to a
validator being slashed are few, but they are severe violations of protocol
rules that may be considered as a potential solo or coordinated attack on the
system. Regardless of the reason for the slashing event, they are all handled
in the same way. There is an initial slashing penalty, which is currently set
at 1 ETH (or $\frac{1}{32}$ of the stake) and this is followed 18 days after
the slashing event by another penalty, known as the correlation penalty. The
purpose of the latter is to penalise what may be a coordinated attack on the
chain. Therefore the correlation penalty takes into account slashings 18 days
before and after the slashing event. 

A validator gets slashed when they are reported and evidence of the violation
of the rules is included in a beacon block. For the valid reporting of a
slashing event, the reporting validator receives a reward. The intention is
that this will help incentivise the reporting of slashable events. Only one
proposer slashing can be included in a report, whereas multiple attestation
violations can be included in a report. When the slashing is included in a
block, the proposer gets a reward which is a fraction of the effective balance
of the validator being slashed (currently $\frac{1}{512}$). Up to 16 proposer
slashings can be included in a block and up to 2 attester slashing reports. 
Therefore, if several slashings have occurred, including these reports in a
block can generate a generous reward for the proposer. 

For more detailed information on slashing and calculations, please refer to the 
latest version of Edgington's
\href{https://eth2book.info/capella/part2/incentives/slashing/}{\textit{Upgrading Ethereum}}
book \cite{Edgington2023}  which incorporates the updates included in the 
Capella hard fork. In summary the events that lead to slashing are:

\begin{enumerate}
  \item ``making two differing attestations for the same target checkpoint''
  \item ``making an attestation whose source and target votes `surround' those
    in another attestation from the same validator.
  \item ``proposing more than one distinct block at the same height''
  \item ``attesting to different head blocks, with the same source and target
    checkpoints''
\end{enumerate}

The first two relate to Casper \gls{ffg} consensus, and the latter two are
related to \gls{lmd} \gls{ghost} consensus.

Edgington points out that `slashable behaviours relate to ``equivocation'',
which is when a validator contradicts something it previously advertised to the
network'. Hence it is important for validators to ensure that they do not
`accidentally' equivocate. This could theoretically happen as a result of bugs
in client software, but the vast majority of slashings have been due to node
operators running two different nodes using the same validator keys. The reason
may have been to improve uptime, but the risk of slashing is too great compared
to any potential benefit in uptime \cite{Edgington2023}. There was also an
incident where a validator exploited a vulnerability in a relay operator
running mev-boost, an open source proposer-builder separation protocol.
Flashbots posted a 
\href{https://collective.flashbots.net/t/post-mortem-april-3rd-2023-mev-boost-relay-incident-and-related-timing-issue/1540}{detailed post-mortem of the event}.

Apart from the slashing penalties, a slashed validator accrues attestation
penalties until such time as they exit, which is not until $2^{13} \texttt{
}epochs = 8,192 \texttt{ }epochs \approx 36 \texttt{ } days$ after being
slashed. Moreover,  if there is an inactivity leak at the time, the penalties
imposed on slashed validators will be higher. Slashed validators cannot earn
any attestation rewards while waiting to exit. It seems rather odd, but a
slashed validator can still be elected to be the proposer for the next block.
However, their block will be deemed to be invalid. The only duty for which they
could receive a small reward is if they are selected to be in the sync
committee, but the probability of this happening is very small. 

\subsubsection*{Penalties} Slashing is the most severe penalty a validator is
subjected to and as explained they can accrue several extra penalties while
they wait to exit. However, there are a number of smaller, less serious
`misdemeanours', or failure to perform their duties that can result in
penalties for validators. The validator's stake is reduced by the penalty and
the ETH is burnt, thereby reducing net issuance \cite{Edgington2023}. \\

\textit{Attestation penalties}
\begin{itemize}
  \item Missed source and target votes (i.e. missed Casper FFG votes), but no
    penalty for a missed head vote.
  \item Incorrect source vote, then target vote is missed.
  \item Incorrect source or target vote, then head vote is missed.
\end{itemize}

\textit{Sync committee penalties}
\begin{itemize}
  \item Non-participation of a member incurs a penalty equivalent to the reward
    they would have received if it was correct
\end{itemize}

\subsection{Initial slashing penalty}
% -------------------------------------------------
Currently this penalty is proportional to the validator's effective balance,
and this would result in a penalty of 1 ETH if the validator had 32 ETH.
However, if a fully consolidated validator with 2,048 ETH effective balance
would incur an initial penalty of 64 ETH.

One proposal is to make this penalty a constant value of 1 ETH, but ``If we
decide that a constant penalty is insufficient, ...'' it can be adjusted so
that it monotonically increases with effective balance, e.g. by choosing from
the family of polynomials.

\info [inline] {How do the authors propose to check whether the initial penalty
is sufficient?}

The authors demonstrate the increase in initial penalty such that \\
\textit{initial penalty =} $\frac{EB^x}{32}, \texttt{ } x \leqslant 1$ \\

and generate graphs for $x=1, \frac{15}{16}, \frac{7}{8}, \frac{3}{4},
\frac{1}{2}$ with respect to the graph of a constant initial penalty of 1 ETH
\cite{Neuder2023d}.

The authors conclude that visually it appears that $x = \frac{3}{4} \texttt{ }
\frac{7}{8}$ are good choices in terms of balancing the size of the initial
slashing penalty and the risk for a consolidated validator.

\subsection{Correlation penalty}
% --------------------------------------------
This penalty is incurred roughly 18 days (4,096 epochs) after the slashing
event, half way between its exit epoch and the slashing event. The correlation
penalty is important in assessing whether there appears to have been a
coordinated attack on the chain.

The penalty for a \textit{slashed validator,} $ v_j$, is then calculated as
follows by looking at the previous 36 days from this ``half-way'' epoch:
\begin{equation*}
  \begin{split}
& \textit{correlation penalty} = \frac{\left( \sum_{i=1}^n v_i  \right) \cdot  b \cdot  v_j}{T}, \texttt{ } where \\
& v_j = \textit{effective balance of validator j slashed 18 days ago} \\
& v_i = \textit{effective balance of } i^{th} \textit{ validator slashed during previous 36 days} \\
& n = \textit{number of validators that were slashed during last 36 days} \\
& b =3 \textit{ (multiplier changed to 3 in Bellatrix, from 1  in Phase 0, and 2 in Altair)} \\
& T = \textit{total effective balance of the beacon chain}
  \end{split}
\end{equation*}

For simplicity this is represented as:

\begin{equation*}
\begin{split}
& \textit{correlation penalty (penalty)} = \frac{3\cdot EB\cdot SB}{TB}, \texttt{ } where \\
& 3 = \textit{Bellatrix multiplier} \\
& EB = \textit{slashed validator's effective balance} \\
& SB = \textit{total slashable balance} \\
& TB = \textit{total effective balance of the beacon chain} \\
& \therefore \textit{if SB} = \frac{1}{3} \cdot  TB \implies penalty = EB \\
& \textit{Similarly, if } 3\cdot EB\cdot SB <  TB \implies penalty = 0 \textit{ due to integer division} \\
\end{split}
\end{equation*}

As designed, there is no correlation penalty for isolated slashing events. The
authors point out that this continues to be the case for a fully consolidated
validator with 2,048 ETH effective balance.

They based their calculation on the current staked ETH at the time of writing
which is 24 million ETH.

\begin{equation*}
\begin{split}
& \textit{ SB = EB for an isolated slashing} \\
& \therefore  3\cdot EB\cdot SB =  3\cdot EB\cdot EB \\
& \textit{Assuming EB = 2,048 \& TB } = 2.4 \cdot  10^7, \textit{ then} \\
& 3\cdot EB\cdot EB = 1.2582912 \cdot  10^7 < 2.4 \cdot  10^7 \\
& \therefore penalty = 0
\end{split}
\end{equation*}

The authors include two graphs to demonstrate how the correlation penalty
increases for solo (32 ETH), partially consolidated (256 ETH) and fully
consolidated (2,048 ETH) validators. It is important to ensure that any
modifications to the function for calculating the correlation penalty satisfies
the requirement that when the total slashed balance is $\frac{1}{3}^{rd}$ of
the total balance, the entire balance of the validator is slashed.

The authors propose a function that preserves this requirement if the MaxEB
proposal is implemented:

\begin{equation*}
  \begin{split}
& penalty' = \frac{3^2 \cdot  EB \cdot  SB^2}{TB^2} \\
& \therefore \textit{if SB } = \frac{TB}{3}, \textit{ then} \\
& penalty' =   \frac{3^2 \cdot  EB \cdot  \left(  \frac{TB}{3}^2 \right) }{TB^2} \\
& \therefore penalty' = EB
  \end{split}
\end{equation*}

Moreover, the new correlation penalty function scales quadratically as opposed
to the current function that scales linearly. Clearly with the proposed new
function, the slashing penalties are substantially reduced, not only for fully
consolidated validator, but also for partially consolidated and solo
validators. The authors demonstrate the comparative correlated slashing
penalties for these three types of validators using both the current and the
proposed functions \cite{Neuder2023d}.

% -----------------------------------------
\subsection{Attestation penalty}
% -----------------------------------------
Once a validator is slashed, their attestations (source, target and head votes)
are deemed to be invalid and hence they incur attestation penalties for the
8,192 epochs until their exit epoch. 

Different weights are attached to each vote, but only the source (weight = 14)
and target (weight = 26) votes incur penalties. For each of the 8,192 epochs
the slashed validator will incur:

\begin{equation*}
\begin{split}
& Given:\\
& \textit{base reward} = \frac{64}{ \left\lfloor \sqrt{TB} \right\rfloor} \textit{, weight denominator} = 64, \\
& \textit{source weight} = 14 \textit{  \& target weight} = 26\\
& \\
& \therefore \textit{epoch attestation penalty} = \frac{\textit{base reward} \cdot  EB \cdot  (14 + 26)}{64} \\
& \therefore \textit{ if } TB \approx 24 \textit{million ETH} = 2.4 \cdot  10^6 \cdot  10^9 \textit{ Gwei}\\
& \textit{the integer square root of } 2.4 \cdot  10^6 \textit{ ETH} = 4,898 \textit{, and} \\
& \textit{the integer square root of } 2.4 \cdot  10^6 \cdot  10^9 \textit{ Gwei} = 154,919,333 \\
& \textit{base reward} = \frac{64 \cdot  10^9}{154,919,333} = 413 \textit{ Gwei} \\
& \\
& \therefore \textit{for a \textbf{solo staker} with 32 ETH:} \\
& \textit{total attestation penalty for 8,092 epochs} = 8192 \cdot  \frac{413\cdot 32\cdot 40}{64}  \textit{ Gwei}\\
& \therefore \textit{total attestation penality} \approx 6.767 \cdot  10^7 \textit{ Gwei} \approx 0.06767 \texttt{ } ETH \\
& \\
& \textit{for a \textbf{fully consolidated} validator with 2,048 ETH:} \\
& \textit{total attestation penalty for 8,092 epochs} = 8192 \cdot  \frac{413\cdot 2048\cdot 40}{64}\approx 4.331 \texttt{ } ETH \\
\end{split}
\end{equation*}

This attestation penalty for a large slashed staker seems acceptable, but could
potentially be adjusted by changing the number of epochs that the validator is
deemed as being ``offline''.The size of this penalty needs to be such that the
security model is not compromised. In other words it should never be a better
option to self-slash to avoid inactivity penalties. Therefore, it needs to be
greater than the inactivity penalties for an unslashed validator that is
exiting and offline \cite{Neuder2023d}.

\subsection{Inactivity leak penalty}
% -------------------------------------------------------------------
An inactivity leak is currently defined as the situation when the chain has not
been finalising for 4 epochs (this value is set by the protocol). Online
validators are not penalised when this happens, i.e. no rewards are earned but
the penalty is 0. On the other hand, offline validators, which includes slashed
validators waiting to exit, start `leaking' state. The loss of stake means that
the relative weight of the online validators will increase, which helps the
chain to start finalising again. The inactivity penalty can be quite severe. 

Using the current method of calculating inactivity leaks, the authors worked
out the penalty for validators with three different effective balances: 32 ETH
(solo validator), 256 ETH (partially consolidated validator),  and 2,048 ETH
(fully consolidated validator).

\begin{table}[htp]
\caption{inactivity leak penalties}
\begin{center}
\renewcommand{\arraystretch}{1.3}
\begin{tabular}{|l|l|l|l|}
\hline
\textbf{validator size} & \textbf{16 epoch leak} & \textbf{128 epoch leak} & \textbf{1024 epoch leak} \\
\hline
32 ETH & 0.000259 ETH & 0.0157 ETH & 1.00 ETH \\
256 ETH & 0.00208 ETH & 0.126 ETH & 8.01 ETH \\
2048 ETH & 0.0166 ETH & 1.01 ETH & 64.1 ETH \\
\hline
\end{tabular}
\end{center}
\label{default}
\end{table}%

% ---------------------------------------------------------------
\subsection{Griefing/discouragement attacks}
% ---------------------------------------------------------------
According to Buterin a griefing attack is when a validator acts maliciously
inside a consensus mechanism to reduce other validators' revenue even at some
cost to themselves to encourage the victims to drop out of the mechanism
\cite{buterin2018c}.

The two main motivations for reducing the number of participants are most
likely because fewer participants:
\begin{itemize}
  \item mean greater rewards for those remaining in the mechanism
  \item helps to prepare an attack on the chain by reducing the cost of an
    attack
\end{itemize}

Some strategies have already been put in place to avoid discouragement attacks
\cite{Edgington2023}:
\begin{itemize}
  \item inverse square root scaling of validator rewards
  \item scaling of rewards with participation (viz. for each ``source, target,
    and head vote, the attester's reward is scaled by the proportion of the
    total stake that made the same vote'')
  \item zeroing attestation rewards during an inactivity leak
  \item rate limiting of validator exists, which means that an attacker needs
    to sustain an attack for longer and at greater cost in order to achieve the
    same outcome.
\end{itemize}
\section{Bibliography}
% -----------------------------
\nocite{*}
\bibliographystyle{eptcs}
\bibliography{references}

\end{document}
